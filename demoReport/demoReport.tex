\documentclass[a4paper,11pt]{article}
\newcommand{\todo}[1]{{\tt $\ldots$ #1 $\ldots$ }}
\usepackage[utf8]{inputenc}  % Linux, macOS: enable non-English characters
%\usepackage[latin1]{inputenc}    % Windows: enable non-English characters
\usepackage[british]{babel}
\usepackage{fullpage}

\usepackage{amsmath}
\usepackage{amssymb}
\usepackage{amsthm} \newtheorem{theorem}{Theorem}
\usepackage{booktabs}
%\usepackage{clrscode3e}
\usepackage{color}
\usepackage{courier} % \texttt{...} gives thinner text
\usepackage{float}
\usepackage{mathtools} \mathtoolsset{showonlyrefs}  % provides \coloneqq
\usepackage{multirow}
\usepackage{tikz} \usetikzlibrary{trees}
\usepackage{listings}
\lstset{language=Python,
	basicstyle=\footnotesize\ttfamily,
	keywordstyle=\color[rgb]{0,0.5,0}\ttfamily,
	stringstyle=\color{orange}\ttfamily,
	commentstyle=\color{red}\ttfamily,
	tabsize=2,
	numbers=left, numberstyle=\tiny, numbersep=5pt,
	breaklines=true,
	breakatwhitespace=true,
	prebreak={/},
	captionpos=b,
	columns=fullflexible,
	escapeinside={\#*}{\^^M},
	mathescape
}
\usepackage{hyperref}  % should always be the last package


% useful wrappers for algorithmic/Python notation:
\newcommand{\length}[1]{\text{len}(#1)}
\newcommand{\twodots}{\mathinner{\ldotp\ldotp}}  % taken from clrscode3e.sty
\newcommand{\Returns}{\longrightarrow}

%% useful (wrappers for) math symbols:
\newcommand{\Cardinality}[1]{\left\lvert#1\right\rvert}
%\newcommand{\Cardinality}[1]{\##1}
\newcommand{\Ceiling}[1]{\left\lceil#1\right\rceil}
\newcommand{\Floor}[1]{\left\lfloor#1\right\rfloor}
\newcommand{\Iff}{\Leftrightarrow}
\newcommand{\Implies}{\Rightarrow}
\newcommand{\Intersect}{\cap}
\newcommand{\Oh}[1]{\mathcal{O}\left(#1\right)}
\newcommand{\Sequence}[1]{\left[#1\right]}
\newcommand{\Set}[1]{\left\{#1\right\}}
\newcommand{\SetComp}[2]{\Set{#1\SuchThat#2}}
\newcommand{\SuchThat}{\mid}
\newcommand{\Tuple}[1]{\langle#1\rangle}
\newcommand{\Union}{\cup}

%% fancy characters:
\usepackage{pifont}
%\newcommand{\scissors}{\ding{36}}
%\newcommand{\phone}{\ding{37}}
%\newcommand{\aircraft}{\ding{40}}
%\newcommand{\envelope}{\ding{41}}
\newcommand{\handpoint}{\ding{43}}
%\newcommand{\victory}{\ding{44}}
%\newcommand{\handwrite}{\ding{45}}
%\newcommand{\tick}{\ding{51}}
%\newcommand{\notick}{\ding{55}}

\renewcommand{\thepart}{\arabic{part}}
\renewcommand{\thesection}{\Alph{section}}

\title{\textbf{Algorithms \& Data Structures II (course 1DL231) \\
    Uppsala University -- Autumn 2019 \\
    Report for Assignment n  %% Replace n by 1, 2, or 3
    by Team t}}              %% Replace t by your team number

%% Replace by your name(s) and choose the encoding of line 3 or line 4:
\author{Clara CLÄVER \and Whiz KIDD}

%\date{Month Day, Year}
\date{\today}

\begin{document}

\maketitle

%% Delete this paragraph:
\noindent
This document shows the ingredients of a good assignment report for
this course.  The \LaTeX\ source code of this document illustrates
almost everything you need to know about \LaTeX\ in order to typeset a
professional-looking assignment report (for this course).  Use it as a
starting point for imitation and delete everything irrelevant.  The
usage of \LaTeX\ is \emph{optional}, but highly recommended, for
reasons that will soon become clear to those who have never used it
before; any learning time is \emph{outside} the budget of this course,
but will hugely pay off, if not in this course then in the next
course(s) you take and when writing a thesis or other scientific
report.

\part{Insertion Sort}\label{part:isort}
\setcounter{section}{0} % reset the section counter for each part/problem

\textsc{Insertion-Sort}\footnote{Your report need not contain an
  explanation, like in this paragraph, of the problem to be solved:
  you can start with the task answers, assuming the reader has read
  the problem statement in the assignment.} ``is an efficient
algorithm for sorting a small number of elements.  Insertion sort
works the way many people sort a hand of playing cards.  We start with
an empty left hand and the cards face down on the table.  We then
remove one card at a time from the table and insert it into the
correct position in the left hand.  To find the correct position for a
card, we compare it with each of the cards already in the hand, from
right to left, as illustrated in Figure~\ref{fig:isort}.  At all
times, the cards held in the left hand are sorted, and these cards
were originally the top cards of the pile on the table.'' (Quoted from
page~17 of CLRS3~\cite{CLRS}.)

\begin{figure}[t]  % make it float to the top of a page
  \centering
  \includegraphics[height=6cm]{isort}
  \caption{Sorting a hand of cards using insertion sort.
    (\copyright\ nobody, 2010)}
  \label{fig:isort}
\end{figure}

\medskip

In the sequel of part~\ref{part:isort}, assume that the problem tasks in
the assignment were as follows (actual assignment statements in this
course may have other tasks):
\begin{enumerate}
\renewcommand{\theenumi}{\Alph{enumi}}
\item Implement \textsc{Insertion-Sort} as a Python function
  $\mathit{insertion\_sort}(A)$, where the elements to be sorted are
  provided in an integer array $A$ indexed from $0$ to $n-1$.  The
  sorting is to be done \emph{in place}, returning $A$ in
  non-decreasing order, that is:
  $A[0] \leq A[1] \leq \dots \leq A[n-1]$.  For brevity, you can refer
  to a sequence in non-decreasing order as a \emph{sorted sequence}.
\item Compute the best, average, and worst-case time complexities of
  \textsc{Insertion-Sort}.
\end{enumerate}

\section{Specification and Program}
\label{sec:pgm:isort}

A specification of sorting and our Python implementation of
\textsc{Insertion-Sort} are given in Listing~\ref{pgm:isort}, where
$n$ is referred to as $\length{A}$.

\lstinputlisting[firstline=11,lastline=30,firstnumber=11,
  %% See previous line for how only to include the actual contribution
  float=t,label=pgm:isort,caption={Python
  implementation of the \textsc{Insertion-Sort} algorithm on page~18
  of CLRS3~\cite{CLRS}. \handpoint~Compare for example
  line~\ref{line:ins-sort-while} with line~5 of that algorithm: arrays
  are indexed from~$1$ in CLRS3 but from~$0$ in Python.  Note also
  that a \emph{closed} interval $\ell \twodots u$, as used in the
  mathematical notation of the comments, is denoted by the
  \emph{right-open} interval $\ell : u+1$ in Python; you can use the
  Python notation in comments and the running text, as long as you
  comply with the Python semantics.}]{insertion_sort.py}

\section{Complexity Analysis}
\label{sec:analysis:isort}

The program in Listing~\ref{pgm:isort} has two nested loops, so we
analyse it starting from the inner loop, in
lines~\ref{line:ins-sort-while} to~\ref{line:ins-sort-while-end},
whose purpose is to insert $A[j]$ into the \emph{sorted} sequence
$A[0 \twodots j-1]$, assuming $j>0$, yielding the sorted sequence
$A[0 \twodots j]$.  Let $T_{\text{ins}}(j)$ denote the running time of
this inner loop:
\begin{equation} \label{rec:ins}
  T_{\text{ins}}(j) =
  \begin{cases}
    \Theta(1) & \text{if~} A[j-1] \leq A[j] \hfill \text{(best case)} \\
    \Theta(j) & \text{if~} A\left[\frac{j-1}{2}\right] < A[j] \leq
    A\left[\frac{j+1}{2}\right] \hfill ~~ (\text{if~} j>1) ~~~ \text{(average case)} \\
    \Theta(j) & \text{if~} A[j] < A[0] \hfill \text{(worst case)}
  \end{cases}
\end{equation}
assuming that every comparison takes constant time and every
assignment takes constant time.

We can now analyse the outer loop, and hence the whole algorithm.  Let
$n$ denote $\length{A}$ and let $T(n)$ denote the running time of
$\text{insertion\_sort}(A)$.  We get the following recurrence:
\begin{equation*}
  T(n) =
  \begin{cases}
    \Theta(1) & \text{if~} n < 2 \\
    T(n-1) + T_{\text{ins}}(n) & \text{if~} n \geq 2
  \end{cases}
\end{equation*}
Using recurrence~(\ref{rec:ins}), we get the following time complexity
results:
\begin{itemize}
\item $T(n) = \Theta(n)$ in the \emph{best case}, where the array is
  already non-decreasingly ordered before the sorting, so that
  $T_{\text{ins}}(n) = \Theta(1)$ at \emph{every} iteration of the
  outer loop, because $A[j]$ is always kept by the inner loop
  \emph{behind} the sorted sequence $A[0 \twodots j-1]$.  This result
  follows from Theorem~\ref{thm:ind} below, for the constants $a=1$
  and $b=2$.
\item $T(n) = \Theta(n^2)$ in the \emph{average case}, defined here as
  follows: \emph{on average} over the iterations of the outer loop,
  the inner loop inserts $A[j]$ into the \emph{middle} of the sorted
  sequence $A[0 \twodots j-1]$, so that $T_{\text{ins}}(n) =
  \Theta(n)$ on average at \emph{every} iteration of the outer loop.
  This can be proven by induction: $\langle$insert your proof
  here$\rangle$.
\item $T(n) = \Theta(n^2)$ in the \emph{worst case}, where the array
  is non-\emph{increasingly} ordered before the execution of the
  algorithm, so that $T_{\text{ins}}(n) = \Theta(n)$ at \emph{every}
  iteration of the outer loop, because $A[j]$ is always inserted by
  the inner loop at the \emph{beginning} of the sorted sequence $A[0
  \twodots j-1]$.  This result has the same proof as in the average
  case above.
\end{itemize}
In conclusion, \textsc{Insertion-Sort} takes $\Oh{n^2}$ time for an
array of $n$ elements.

\begin{theorem}
  \label{thm:ind}
  The following recurrence, for some \emph{constants} $a$ and $b$:
  \[
  T(n) =
  \begin{cases}
    \Theta(1) & \text{if~} n < b \\
    a \cdot T(n-1) + \Theta(1) & \text{if~} n \geq b
  \end{cases}
  \]
  has $\Theta(n)$ as closed form for $a = 1$, and $\Theta(a^n)$ as
  closed form for $a > 1$.
\end{theorem}

\begin{proof}
  By induction (left as an exercise to the reader in the AD1 course).
\end{proof}

\part{Weighted Interval Scheduling}\label{part:dp}
\setcounter{section}{0} % reset the section counter for each part/problem

\begin{quote}
``Suppose we have a set $S = \Set{a_1, a_2, \dots, a_n}$ of $n$ proposed
\emph{activities} that wish to use a resource, such as a lecture hall,
which can serve only one activity at a time. Each activity $a_i$ has a
\emph{start time} $s_i$ and a finish time $f_i$ , where
$0 \leq s_i < f_i < \infty$. If selected, activity $a_i$ takes place
during the half-open time interval $[s_i, f_i)$. Activities $a_i$ and
$a_j$ are \emph{compatible} if the intervals $[s_i, f_i)$ and
$[s_j, f_j)$ do not overlap. That is, $a_i$ and $a_j$ are compatible if
$s_i \geq f_j$ or $s_j \geq f_i$. In the 
\emph{activity-selection problem}, we wish to select a maximum-size
subset of mutually compatible activities.''
\end{quote}
(Quoted from page~415 of
CLRS3~\cite{CLRS}.)

The weighted interval scheduling problem is an extension to the 
activity-selection problem where activity $a_i$ has the weight $w_i$ and
where an optimal solution is to be found. An optimal solution to the
weighted interval problem is a subset,
$O \subseteq \Set{a_1, a_2, \dots, a_n}$, where the total weights over
the selected activities, $\sum \SetComp{w_i}{a_i \in O}$, is maximised.

Additionally, the activities are sorted in non-decreasing order of
finish times: $f_1 \leq f_2 \leq \cdots \leq f_n$ and the help function
$p(i)$ gives the largest index $j < i$ such that activity $i$ and $j$ are
compatible:

\begin{equation} \label{dp:helpfun}
  p(i) = 
  \begin{cases}
	\mathop{\mathrm{arg\,max}}\limits_{j\in 1 \twodots i-1} ( s_i \geq f_j ) & \hfill \qquad \text{if } \exists j \in 1\twodots i-1 : s_i \geq f_j \\
	0 & \hfill \text{otherwise}
  \end{cases}
\end{equation}

\medskip

In the sequel of part~\ref{part:dp}, assume that the problem tasks in
the assignment were as follows (actual assignment statements in this
course may have other tasks):
\begin{enumerate}
\renewcommand{\theenumi}{\Alph{enumi}}
\item Give a recursive equation for a parameterised quantity after 
  stating its meaning in terms of all its parameters. Use the equation
  to justify that the problem has the optimal substructure property
  and overlapping subproblems, so that dynamic programming is applicable
  to it.
\item Motivate your choice between bottom-up iteration (for which you
  must argue for  the chosen nesting and iteration order of the loops)
  and top-down recursion for the Weighted Interval Scheduling Problem.
\end{enumerate}

\section{Recursive equation}
\label{sec:recursive:actsel}

The weighted interval scheduling problem can be represented by the following
recursive function:

\begin{equation} \label{dp:opt}
  OPT(i) =
  \begin{cases}
    0 & \hfill \text{if } i = 0 \\
    \max~\{~OPT(i - 1), w_i + OPT(p(i))~\} & \hfill \text{if } i > 0
  \end{cases}
\end{equation}

Given an instance of the Weighted Interval Scheduling problem with an optimal
solution $O$, $O \subseteq J$. The last activity, $a_n$, either belongs to
$O$, $a_n \in O$, or it does not, $a_n \notin O$. If $a_n$ belongs to $O$,
then all intervals not compatible with $a_n$,
$\SetComp{a_i}{i \in p(n)+1 \twodots n-1}$, do not belong to $O$.

Additionally, if $a_n$ belongs to $O$, then $O$ must include an optimal
solution, $O^\prime_{p(n)}$, to the subproblem
$\SetComp{a_i}{i \in 1 \twodots p(n)}$. If $O^\prime_{p(n)}$ is not optimal,
then $O^\prime_{p(n)}$ could be modified into a solution that is optimal
and where all activities are intrinsically compatible with $a_n$.

If $a_n$ does not belong in $O$, then $O$ is an optimal solution to the
subproblem $\SetComp{a_i}{i \in 1 \twodots n-1}$. The reasoning is
analogous: assume that $a_n \notin O$; so if $O$ is not an
optimal solution to the subproblem of activities
$\SetComp{a_i}{a_i \in 1 \twodots n-1}$, then $O$ could be modified into
a solution that is.

Deciding if $a_n$ is to belong in $O$ thus require an optimal solution to the
subproblems $\SetComp{a_i}{i \in 1\twodots j}$, with $j$ taking the value
between $1$ and $n$, $j \in 1 \twodots n$, proving that Weighted Interval
scheduling has optimal substructure.

Given the index $i$ of an activity $a_i$, the recursive equation~\ref{dp:opt}
returns the weight $0$ for index $0$, denoting an empty set of activities,
and otherwise the optimal solution to the two subproblems: when activity $a_i$ does
not belong to the optimal subproblem for the activities 
$\SetComp{a_k}{k \in 1 \twodots i-1}$, or the optimal solution to the
subproblem with activities $\SetComp{a_k}{k \in 1 \twodots p(i)}$ plus the additional
weight $w_i$.

There are some problem instances that have overlapping subproblems. For
example; given the activities $\Set{a_1, a_2, \dots, a_n}$, $p(i)=k$, and $k\geq 1$,
then the subproblem $\Set{a_1, a_2, \dots, a_k}$ will be calculated at least
two times, once when finding the optimal solution for the problem $OPT(n)$ and
once for the subproblem $OPT(k+1)$.

\section{Bottom-Up Iteration and Top-Down Recursion}
\label{sec:bottom-up-rec}

Given the reasoning and the recursive function in task~\ref{sec:recursive:actsel},
the optimal solution to each subproblem $S_i$, with $1 \leq i \leq n$
and $S_i=\Set{a_1, \dots, a_i}$, will be found, giving
bottom-up iteration and top-down recursion the same time complexity.
We have chosen the bottom-up iterative approach. Our program
uses a single loop where all activities are iterated over in
increasing order of their indices, where activity $a_i$ has index $i$.


\bibliographystyle{abbrv}
\bibliography{ad2}  % Add your references to the ad2.bib file

%% Comment away the following line before submitting:
\input{appendix}

\end{document}

%%% Local Variables:
%%% mode: latex
%%% TeX-master: t
%%% End:
